\documentclass{article}
\usepackage[utf8]{inputenc}
\usepackage[brazil]{babel}
\usepackage{setspace}
\usepackage{mathtools}
\usepackage{pgfplots}
\usepackage{listings}
\usepackage{xcolor}
\usepackage{natbib}
\usepackage{graphicx}
\usepackage{hyperref}

\DeclarePairedDelimiter\ceil{\lceil}{\rceil}
\DeclarePairedDelimiter\floor{\lfloor}{\rfloor}

\definecolor{codegreen}{rgb}{0,0.6,0}
\definecolor{codegray}{rgb}{0.5,0.5,0.5}
\definecolor{codepurple}{rgb}{0.58,0,0.82}
\definecolor{backcolour}{rgb}{0.95,0.95,0.92}

\hypersetup{
    colorlinks=true,
    linkcolor=blue,
    filecolor=magenta,      
    urlcolor=cyan,
}


\lstdefinestyle{codigo}{
    numberstyle=\tiny,
    basicstyle=\ttfamily\footnotesize,
    breakatwhitespace=false,         
    breaklines=true,                 
    captionpos=b,                    
    keepspaces=true,                 
    numbers=left,                    
    numbersep=5pt,                  
    showspaces=false,                
    showstringspaces=false,
    showtabs=false,                  
    tabsize=2
}

\lstset{style=codigo}

\setstretch{1.5}
\pgfplotsset{width=10cm,compat=1.9}

\pagenumbering{gobble}
\clearpage
\thispagestyle{empty}

\title{Learning Digital Object - Embarcado veiculares}
\author{Lucas Bottrel \and Lucas Santiago \and Marcelo França \and Miguel Leocádio \and Rafael Amauri}
\date{}



\begin{document}
\maketitle

\newpage

\section*{TRANSMISSÃO DE DADOS}
\subsection*{\underline{Histórico}}

\hspace{4pt} Antes da década de 90, a transmissão de informações entre as Unidades Eletrônicas de Controle (responsáveis por 
interpretar dados e controlar atuadores e outros componentes) era feita utilizando fiação comum. Um carro de luxo 
comum desta época continha, em média, \href{https://www.youtube.com/watch?v=ptH8zxhf-jM}{30 quilos de fiação}, mais de 1 km de fios de cobre e mais de 2 mil conexões separadas. 
Toda essa estrutura era muito cara para fabricar, manter e instalar, além de ser suscetível a erros.


\begin{figure}[ht]
    \centering
    \includegraphics[scale=0.2]{Imagens/Transmissao de Dados/Sistema eletrico de um carro.png}
    \caption{Alguns componentes básicos que são controlados eletricamente. (crédito: \href{https://www.howacarworks.com/basics/how-car-electrical-systems-work}{How A Car Works})}
\end{figure}


\subsection*{\underline{CAN}}

\hspace{4pt} Ao longo dos anos 90, a indústria passou a adotar um novo padrão de rede interna dos veículos desenvolvido pela \href{http://esd.cs.ucr.edu/webres/can20.pdf}{Bosch}: o barramento CAN (Controller Area Network).

Os \href{https://www.youtube.com/watch?v=5eh8Poz5g7M}{barramentos CAN} fazem uso de sinais digitais para enviar dados para as Unidades Eletrônicas de Controle. São bidirecionais, ou seja, os dados podem fluir em ambas as direções, e contam com um sistema elétrico de detecção de falhas e de definição de sinais prioritários. A CAN representou um grande avanço na transmissão de dados em veículos, reduzindo o tamanho, peso e complexidade das redes embarcadas.


\begin{figure}[ht]
    \centering
    \includegraphics[scale=0.6]{Imagens/Transmissao de Dados/CAN bus.png}
    \caption{Estrutura de uma mensagem transmitida por um barramento CAN. (crédito: \href{https://www.typhoon-hil.com/documentation/typhoon-hil-schematic-editor-library/References/can_bus_protocol.html}{Typhoon HIL})}
\end{figure}


\newpage


\subsection*{\underline{Ethernet: o futuro}}

\hspace{4pt} Na atualidade, projetistas estão desenvolvendo uma nova estrutura de rede para os automóveis usando a tecnologia Ethernet (a mesma usada nos famosos “cabos de rede”).

As vantagens de usar um barramento Ethernet residem na maior largura de banda (mais dados transmitidos simultaneamente), sua adaptabilidade (a taxa de transmissão passou dos Megabits para os Gigabits por segundo) e grande número de fornecedores, além de ter a capacidade de se ter uma \href{https://www.youtube.com/watch?v=JTv8p7VFxGE}{mídia física única} para transmissão de quaisquer tipos de dados.

Alex Tan, gerente geral de Soluções Ethernet Automotivas na empresa NXP Semiconductors, afirma que o uso da \href{https://www.youtube.com/watch?v=SpfG-LwO8MQ}{tecnologia Ethernet} adaptada aos automóveis será capaz de gerar melhorias nos veículos sem alterações significativas de hardware, facilitando a integração da estrutura já existente com o recebimento de dados da internet e com novos equipamentos como câmeras e sensores.


\begin{figure}[ht]
    \centering
    \includegraphics[scale=0.4]{Imagens/Transmissao de Dados/Arquitetura de redes veiculares.png}
    \caption{Pesquisas estão sendo desenvolvidas para gerar uma integração no trânsito que vai além do recebimento de dados nos veículos. (crédito: \href{https://www.researchgate.net/figure/Vehicular-network-architecture_fig1_318993172}{ResarchGate})}
\end{figure}


\newpage

\section*{Segurança}
Conhecimento anterior: \href{https://www.youtube.com/watch?v=PLiE0Nr8VOE}{Assista esse vídeo} (o vídeo está em inglês, mas é possível usar a legenda automática do YouTube para português brasileiro).

Assim como dito no vídeo, automóveis são extremamente complicados de se manterem atualizados e qualquer falha de segurança pode ser bastante grave. Por exemplo, se uma falha de segurança permitir que uma pessoa tenha o controle sobre o acelerador e o freio de um carro, quais seriam as implicações na segurança dos passageiros? Para que isso não acontecesse, seria importante atualizações constantes para o software do carro, mas um carro pode durar quase 20 anos, como manteríamos 20 anos de atualização de algum dispositivo? \href{https://baratodecelular.com.br/sac-0800/quanto-tempo-dura-um-celular-android/}{Se um celular hoje dura alguns poucos anos}, como seria o caso de um automóvel?


\begin{figure}[ht]
    \centering
    \includegraphics[scale=1.3]{Imagens/Seguranca/Seguranca IOT.png}
    \caption{Fonte: \href{https://bics.com/iot-safe-robust-iot-security-at-scale/}{Link}}
\end{figure}


\newpage

\section*{SEGURANÇA - Avançado}

\subsection*{Caso se interesse pela área e queira ir mais a fundo entre nesse \href{https://bluepex.com.br/industria-iot/}{site} e segue a imagem abaixo:}


\begin{figure}[ht]
    \centering
    \includegraphics[scale=2]{Imagens/Seguranca/IOT.png}
    \caption{Fonte: \href{https://bluepex.com.br/industria-iot/}{Link}}
\end{figure}


Esse site conta um pouco a história da dificuldade da segurança dos \href{https://pt.wikipedia.org/wiki/Internet_das_coisas#Privacidade_e_seguran%C3%A7a}{IOTs}. Há nele várias perguntas válidas sobre a temática. É importante ressaltar que algumas dessas perguntas não foram respondidas ainda e que estamos estudando métodos melhores de resolvê-las. Carros também usam vários \href{https://pt.wikipedia.org/wiki/Internet_das_coisas#Arduino_e_a_Internet_das_coisas}{IOTs}, como sensores que serão abordados em outro capítulo deste documento. Todas as tecnologias mais novas serão abordadas na foto do próximo nível de segurança:

\newpage

\section*{SEGURANÇA - Caso queira ainda mais:}
\subsection*{Caso queira pesquisar as temáticas de segurança}


\begin{figure}[ht]
    \centering
    \includegraphics[scale=1.4]{Imagens/Seguranca/Mapa mental temas de seguranca.png}
    \caption{Fonte: \href{https://github.com/solid-titans/LDO-Embarcados-Veiculares}{Criada para o projeto}, \href{https://github.com/solid-titans/LDO-Embarcados-Veiculares/blob/main/LICENSE}{mesma licensa do projeto.}}
\end{figure}


\newpage


\section*{SENSORES E HARDWARE}

\hspace{4pt} Com o desenvolvimento de novas tecnologias e aprimoramento do poder de processamento de \href{https://pt.wikipedia.org/wiki/Sistema_embarcado}{sistemas embarcados}, sensores cada vez mais precisos e poderosos puderam ser colocados em veículos para melhorar acessibilidade e conforto do usuário. Um exemplo de sistema embarcado com sensores é um \href{https://www.youtube.com/watch?v=AxapuVgepV8}{sensor de chuva} para para-brisas de carro, ou um \href{https://www.youtube.com/watch?v=EppMSSYBOm0}{sistema para estacionamento automático}. 

Até onde a indústria automobilística vai, um grande número de indústrias e companhias estão envolvidos no desenvolvimento, na manufatura e na venda de carros, motos e ônibus, e a junção dessa indústria com a de tecnologia por meio dos sistemas embarcados é a próxima etapa na evolução  do mercado automobilístico por causa da integração entre o mundo virtual e o real por meio desses aparelhos.

O futuro reserva tecnologias interessantes que já estão sendo desenvolvidas. O \href{https://www.youtube.com/watch?v=j7XKgLtTN2o}{controle automático adaptativo} é uma das mais novas tecnologias que estão sendo aperfeiçoadas. Além de ser usado em veículos de civis, também têm um foco em veículos comerciais, como o \href{https://external-content.duckduckgo.com/iu/?u=https%3A%2F%2Fcdn.cnn.com%2Fcnnnext%2Fdam%2Fassets%2F191121212606-tesla-cybertruck-exlarge-169.jpg&f=1&nofb=1}{Tesla Cybertruck}.


\begin{figure}[ht]
    \centering
    \includegraphics[scale=0.75]{Imagens/Sensores e Hardware/Foto stock nivel hard.png}
    \caption{Fonte: \href{https://github.com/solid-titans/LDO-Embarcados-Veiculares}{Criada para o projeto}, \href{https://github.com/solid-titans/LDO-Embarcados-Veiculares/blob/main/LICENSE}{mesma licensa do projeto.}}
\end{figure}


\newpage

\section*{Caso queira entrar na área de Machine Learning:}

\subsection*{\hspace*{4pt} Há um repositório dedicado à \href{https://github.com/MysteRys337/Trabalho-LDO-ML}{Machine learning}.}

\begin{figure}[ht]
    \centering
    \includegraphics[scale=0.35]{Imagens/Machine Learning/stock.png}
    \caption{Fonte: \href{https://github.com/MysteRys337/Trabalho-LDO-ML}{Projeto LDO de Machine Learning}}
\end{figure}


\end{document}