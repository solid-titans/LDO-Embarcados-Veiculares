\documentclass{article}
\usepackage[utf8]{inputenc}
\usepackage[brazil]{babel}
\usepackage{setspace}
\usepackage{mathtools}
\usepackage{pgfplots}
\usepackage{listings}
\usepackage{xcolor}
\usepackage{natbib}
\usepackage{graphicx}
\usepackage{hyperref}

\DeclarePairedDelimiter\ceil{\lceil}{\rceil}
\DeclarePairedDelimiter\floor{\lfloor}{\rfloor}

\definecolor{codegreen}{rgb}{0,0.6,0}
\definecolor{codegray}{rgb}{0.5,0.5,0.5}
\definecolor{codepurple}{rgb}{0.58,0,0.82}
\definecolor{backcolour}{rgb}{0.95,0.95,0.92}

\hypersetup{
    colorlinks=true,
    linkcolor=blue,
    filecolor=magenta,      
    urlcolor=cyan,
}


\lstdefinestyle{codigo}{
    numberstyle=\tiny,
    basicstyle=\ttfamily\footnotesize,
    breakatwhitespace=false,         
    breaklines=true,                 
    captionpos=b,                    
    keepspaces=true,                 
    numbers=left,                    
    numbersep=5pt,                  
    showspaces=false,                
    showstringspaces=false,
    showtabs=false,                  
    tabsize=2
}

\lstset{style=codigo}

\setstretch{1.5}
\pgfplotsset{width=10cm,compat=1.9}

\pagenumbering{gobble}
\clearpage
\thispagestyle{empty}

\title{Learning Digital Object - Embarcado veiculares}
\author{Lucas Bottrel \and Lucas Santiago \and Marcelo França \and Miguel Leocádio \and Rafael Amauri}



\begin{document}
\maketitle

\newpage

\section*{TRANSMISSÃO DE DADOS}
\subsection*{\underline{Histórico}}

\hspace{4pt} Antes da década de 90, a transmissão de informações entre as Unidades Eletrônicas de Controle (responsáveis por 
interpretar dados e controlar atuadores e outros componentes) era feita utilizando fiação comum. Um carro de luxo 
comum desta época continha, em média, \href{https://www.youtube.com/watch?v=ptH8zxhf-jM}{30 quilos de fiação}, mais de 1 km de fios de cobre e mais de 2 mil conexões separadas. 
Toda essa estrutura era muito cara para fabricar, manter e instalar, além de ser suscetível a erros.


\begin{frame}

    \begin{figure}[ht]
        \centering
        \includegraphics[scale=0.2]{Imagens/Transmissao de Dados/Sistema eletrico de um carro.png}
        \caption{Alguns componentes básicos que são controlados eletricamente. (crédito: \href{https://www.howacarworks.com/basics/how-car-electrical-systems-work}{How A Car Works})}
    \end{figure}

\end{frame}

\subsection*{\underline{CAN}}

\hspace{4pt} Ao longo dos anos 90, a indústria passou a adotar um novo padrão de rede interna dos veículos desenvolvido pela \href{http://esd.cs.ucr.edu/webres/can20.pdf}{Bosch}: o barramento CAN (Controller Area Network).
Os barramentos CAN fazem uso de sinais digitais para enviar dados para as Unidades Eletrônicas de Controle. São bidirecionais, ou seja, os dados podem fluir em ambas as direções, e contam com um sistema elétrico de detecção de falhas e de definição de sinais prioritários. A CAN representou um grande avanço na transmissão de dados em veículos, reduzindo o tamanho, peso e complexidade das redes embarcadas.


\end{document}